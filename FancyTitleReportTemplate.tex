\documentclass[12p]{article}
% \documentclass{seg thigy}
% bibliographystyle{seg asdf}
% \tiny\bibligoraphy{seq_eg} .bib file
% \usepackage[margin=1in, headheight=110pt]{geometry}
\usepackage[letterpaper, margin=1in]{geometry}
\usepackage{amssymb, amsmath, amsfonts, amsthm}
\usepackage{mathpazo}
\usepackage{setspace}
% \usepackage{probsoln}
\usepackage{fancyhdr}
\usepackage{hyperref}
\usepackage{float}
\usepackage{tikz}
\usepackage{enumitem}
\usepackage{listings}
% \usepackage{lipsum}
% \usepackage{parskip} % Use for extra line spacing

\pagestyle{fancy}
\lhead{Evan Quan - 10154242}
\rhead{GOPH 549 - Summer 2017}
\newenvironment{hangingpar}[1]
  {\begin{list}
          {}
          {\setlength{\itemindent}{-#1}%%'
           \setlength{\leftmargin}{#1}%%'
           \setlength{\itemsep}{0pt}%%'
           \setlength{\parsep}{\parskip}%%'
           \setlength{\topsep}{\parskip}%%'
           }
    \setlength{\parindent}{-#1}%%
    \item[]
  }
  {\end{list}}

\newcommand{\sep}{\;}
\newtheorem{theorem}{Theorem}
\newtheorem{lemma}[theorem]{Lemma}
\newtheorem{prop}[theorem]{Proposition}
\newtheorem{cor}[theorem]{Corollary}
\newtheorem{corollary}[theorem]{Corollary}
\theoremstyle{definition}
\newtheorem{definition}[theorem]{Definition}
\newcounter{problem}
\newcounter{exercise}
\newcounter{solution}
\newcounter{question}

\newcommand\floor[1]{\lfloor#1\rfloor}
\newcommand\ceil[1]{\lceil#1\rceil}

\newcommand\Exercise{%
  \stepcounter{exercise}%
  \textbf{Exercise \theexercise}~%
  \setcounter{solution}{0}%
}

\newcommand\Problem{%
  \stepcounter{problem}%
  \textbf{Problem \theproblem}~%
  \setcounter{solution}{0}%
}

\newcommand\Question{%
  \stepcounter{question}%
  \textbf{Question \thequestion}~%
  \setcounter{solution}{0}%
}

\newcommand\ChallengeProblem{%
  \stepcounter{problem}%
  \textbf{Exercise~\theproblem $^{*}$ }~%
  \setcounter{solution}{0}%
}
\newcommand\TheSolution{%
  \textbf{Solution.}~
}
\begin{document}
% \begin{titlepage}
%     \begin{center}
%         \vspace*{1cm}
%
%         \textsc{\Large Wenner, Dipole-Dipole, and Seismic Joint Inversion}
%         \vspace{0.3cm}
%
%         \vspace{2cm}
%
%         {Evan Quan}
%
%         \vspace{2cm}
%         Date of submission: September 29, 2017
%     \end{center}
% \end{titlepage}
\begin{titlepage}
  \begin{center}
    \vspace*{1cm}
    \Large{\textbf{University of Calgary}}\\
    \Large{\textbf{GOPH 549 - Field School}\\
    \vfill
    \line(1,0){400}\\[1mm]
    \huge{\textbf{Wenner, Dipole-Dipole,\\
    and Seismic Joint Inversion}}\\
    % \Large{\textbf{A Study on Forward-Modelling and Inversion}}\\
    \line(1,0){400}\\
    \vfill
    Evan Quan\\
    ID: 10154242\\
    September 29, 2017
    % \today \\
  \end{center}
\end{titlepage}
% \thispagestyle{fancy}
\setcounter{page}{0}
\tableofcontents
\pagenumbering{gobble}
\break
\pagenumbering{arabic}
% \onehalfspacing

\section{Section 1}


\section{References}
\begin{hangingpar}{2em}
  Bigiarini, Mauricio Zambrano, 2017, Normalized Root Mean Square Error.\\
  https://www.rforge.net/doc/packages/hydroGOF/nrmse.html, accessed 19 September 2017.

  Holmes, Susan, 2000, RMS Error.\\
  http://statweb.stanford.edu/~susan/courses/s60/split/node60.html, accessed 19 September 2017.

  Inanen, Kris. Refraction methods. Class lecture, University of Calgary, Calgary, Canada, February 2016.

  Laurer, Rachel. Introduction to Electrical Resistivity: Wenner Array. Class lab, University of Calgary, Calgary, Canada, 2016.

  Laurer, Rachel. Resistivity. Class lecture, University of Calgary, Calgary, Canada, March 2017.

  Telford, William Murray, L.P. Geldart, R.E. Sheriff, 1990, Applied Geophysics, Second Edition: Cambridge University Press.
\end{hangingpar}
% ERT multilayer equation Telford???
\end{document}
